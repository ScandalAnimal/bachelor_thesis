\chapter{Úvod}
Booleova algebra má značné využitie vo viacerých oblastiach vedy. Jej základným a dnes hlavným využitím je binárna reprezentácia stavov tranzistorov v počítačovej vede, a tým pádov využitie jediného bitu. Okrem toho ale svoje využitie nachádza aj pri návrhu číslicových obvodov ako efektívna reprezentácia chovania jednotlivých harvérových komponentov, v teórií grafov pre návrh orientovaných grafov, či v klasickom high-level programovaní ako vyjadrenie rôznych stavov systému.

Bohaté využitie má takisto aj v matematike vo výrokovej logike a kombinatorike. Uplatnenie booleovej algebry je možné vidieť aj v oblasti umelej inteligencie, teórie mechanického učenia a teórie hier. Z netechnických odborov stojí za zmienku oblasť legislatívy, kde sa využíva booleova logika napríklad pri voľbách do štátnych funkcií.

Existujú viaceré reprezentácie booleovskych funkcií, ktoré sa líšia svojím využitíme. Klasické reprezentácie formou pravdivostných tabuliek nachádzajú svoje využitie v matematike, ale pre informatiku nie sú vhodné. V priebehu času boli vyvinuté rôzne metódy pre symbolizáciu týchto funkcií v počítačovom programe, z nich najpoužívanejšia je reprezentácia binárnymi rozhodovacími diagramami (skrátene BDD z anglického Binary Decision Diagram). Jednou z výhod reprezentácie pomocou BDD je, že dokážu vytvoriť kanonickú formu funkcie. BDD umožňujú veľmi dobre zisťovať ekvivalenciu a splniteľnosť booleovskych funkcií.

Reprezentácia pomocou BDD v informatike je síce najrozšírenejšia, ale booleovske funkcie sa dajú reprezentovať aj inou formou. V tejto práci sa budeme zaoberať reprezentáciou boolevskych funkcií pomocou algebraickej normálnej formy (skrátene ANF). ANF poskytuje výhodu oproti BDD v tom, že obsahuje len operácie AND a XOR, a tým pádom sa jej implementácia značne zjednodušuje. Takisto je z ANF možné rýchlo vyčítať hodnotu danej funkcie, a takisto vypočítať jej splniteľnosť v rozumnom čase. 

Vytvorená knižnica poskytuje prostriedky pre efektívnu manipuláciu a zobrazovanie booleovskych funkcií v ANF. Motiváciou pre vytvorenie knižnice bolo vytvoriť slušnú alternatívu pre klasické reprezentácie pomocou BDD pre špecifické problémy, ktoré nepotrebujú komplexnuú reprezentáciu BDD, ale vystačia si aj s ANF.

V tejto práci si v kapitole ?? povieme najskôr niečo teoreticky o rôznych reprezentáciách booleovskych funkcií, o ich výhodách a necýhodách. V kapitole ?? si odprezentujeme existujúce knižnice a ich vzužitie v praxi. V kapitola ?? sa budeme zaoberať technickým návrhom knižnice, v kapitole ?? jej konkrétnou implementáciou. Na záver si v kapitole ?? porovnáme vytvorenú knižnicu s existujúcimi a vyvodíme z toho závery. 
 

\chapter{Booleovske funkcie}
V tejto kapitole sa nachádza teoretický úvod do problematiky booleovskych funkcií, postupne bude definované čo vlastne sú booleovske funkcie, čo sa dá pomocou nich popísať a aký môže byť ich obsah. Ďalej budú popísané rôzne možnosti zobrazenia booleovskych funkcií napríklad pravdivostné tabuľky a ďalšie. Kapitola takisto definuje rôzne normalizované formy zápisu booleovskych funkcií, pričom dôraz bude kladený hlavne na algebraickú normálnu formu, ktorej reprezentácia je cieľom celej práce. Podrobnejšie bude popísaná aj reprezentácia binárnymi rozhodovacími diagramami, ktoré sú momentálne najpoužívanejšou reprezentáciou v oblasti počítačovej vedy.

\section{Definícia booleovskej funkcie}
Ako uvádza Crama \cite{Crama-bool}, booleovská funkcia je každá funkcia $f: \mathcal{B}^{n} \rightarrow \mathcal{B}$, kde $\mathcal{B}$ je množina $\{0,1\}$, v ktorej $n$ je kladné prirodzené číslo, a $\mathcal{B}^{n}$ označuje $n$-násobný kartézsky súčin množiny $\mathcal{B}$ samej so sebou. Každý bod funkcie $X^*$ = ($x_1,x_2, \ldots, x_n$) naberá hodnotu buď 0 alebo 1 z množiny $\mathcal{B}$.

Celkový počet rôznych booleovskych funkcií pre $n$ premenných je $2^{2^n}$. Je to dané tým, že všetkých možných kombinácií vstupných parametrov je ($2^n$) a parametre môžu mať hodnotu z $\{0,1\}$. Tento počet obsahuje aj kombináciu o 0 prvkoch, takže sa častejšie uvádza číslo $2^{2^n-1}$. Počet možných booleovských funkcií pre niektoré hodnoty $n$ sa nachádza v Tabuľke {\ref{table:functionCountExample}}. Je vidieť že počet možných kombinácií prudko narastá s počtom premenných, a teda efektívna reprezentácia je nutnosťou.

\begin{table}[h]
	\centering
	\begin{tabular}{|r|l|} \hline
		n & počet funkcií \\ \hline
		1 & 4   \\
		2 & 16  \\
		3 & 256 \\
		5 & 4.29497 $\times 10^9$ \\ 
		6 & 1.84467 $\times 10^{19}$ \\ \hline
	\end{tabular}
	\caption{Počet booleovských funkcií pre vybrané hodnoty $n$}
	\label{table:functionCountExample}
\end{table}

V mnohých aplikáciách sa pre predstavu hodnôt množiny $\mathcal{B}$ namiesto dvojice \{0,1\} používa iná dvojica, napríklad \{true,false\}, \{1,-1\}, \{on,off\}, \{áno,nie\}, vždy to ale označuje opačné hodnoty. 
Množina $\mathcal{B}$ spolu so základnými booleovskymi operáciami konjunkciou $\wedge$, disjunkciou $\vee$ a negáciou $\neg$ tvorí Booleovsku algebru. Booleovskú algebru tvorí niekoľko základných pravidiel, ktoré sú popísané v Tabuľke \ref{table:boolAlgebra}.

\begin{table}[h]
	\centering
	\begin{tabular}{|l l|} \hline
		asociativita & $(x \vee y) \vee z = x \vee (y \vee z)$ \\
		& $(x \wedge y) \wedge z = x \wedge (y \wedge z)$ \\
		komutativita &  $x \vee y = y \vee x$ \\
		& $x \wedge y = y \wedge x$ \\
		absorpcia & $x \vee (x \wedge y) = x$ \\
		& $x \wedge (x \vee y) = x$ \\
		distributívnosť & $x \vee (y \wedge z) = (x \vee y) \wedge (x \vee z)$ \\
		& $x \wedge (y \vee z) = x \wedge y \vee x \wedge z$ \\
		komplementarita & $x \vee \neg x = 1$ \\
		&  $x \wedge \neg x = 0$ \\  
		agresivita nuly & $x \wedge 0 = 0$ \\
		agresivita jednotky & $x \vee 1 = 1$ \\
		idempotencia & $x \vee x = x$ \\
		& $x \wedge x = x$ \\ 
		absorpcia negácie & $x \vee (\neg x \wedge y) = x \vee y$ \\
		& $x \wedge (\neg x \vee y) = x \wedge y$ \\ 
		dvojitá negácia & $\neg (\neg x) = x$ \\ 
		De Morganove zákony & $\neg x \wedge \neg y = \neg (x \vee y)$ \\
		& $\neg x \vee \neg y = \neg (x \wedge y)$ \\ \hline
	\end{tabular}
	\caption{Pravidlá Boolovskej algebry}
	\label{table:boolAlgebra}
\end{table}

\section{Reprezentácia booleovskych funkcií}
Booleovske funkcie môžu byť vyjadrené rôznymi spôsobmi. Záleží hlavne na tom, čo plánujeme s danou funkciou robiť. Niektoré zápisy sú vhodnejšie na matematické výpočty, iné zase na prehľadné prezeranie dát. 

\vspace*{0.5\baselineskip}
Prvým možným zápisom je pravdivostná tabuľka. Je to tabuľka, v ktorej na každom riadku je hodnota funkcie pri inú kombináciu vstupných hodnôt funkcie. Pravdivostné tabuľky majú dobré využitie pre funkcie do 3-4 parametrov. Pre vyšší počet parametrov sa stávajú neprehľadnými pre vysoký počet možných kombinácií. Príklad pravdivostnej tabuľky pre 2 vstupné hodnoty sa nachádza v Tabuľke \ref{table:truthTable}. 

\begin{table}[h]
	\centering
	\begin{tabular}{|c|c|} \hline
		$(x_1,x_2)$ & $f(x_1,x_2)$ \\ \hline
		$(0,0)$ & 0 \\
		$(0,1)$ & 1 \\
		$(1,0)$ & 1 \\
		$(1,1)$ & 0 \\ \hline
	\end{tabular}
	\caption{Pravdivostná tabuľka}
	\label{table:truthTable}
\end{table}

Upravenou formou pravdivostnej tabuľky je Karnaughova mapa. Je to forma zápisu ktorá prevádza n-rozmernú booleovsku funkciu do 2-rozmernej. Jej výhdou je, že sa pomocou nej dá funkcia pekne vizualizovať, do 5 premenných poskytuje stále dobrú predstavu. Využíva sa hlavne pri minimalizácii funkcií. Je vhodná pre ľudskú predstavu funkcie, pre počítač existujú efektívnejšie alternatívy. Príklad Karnaughovej mapy sa nachádza na Obrázku \ref{picture:KarnaughMap}.


Ďalším zo zápisov je logický obvod. Ide o schému, ktorá graficky zobrazuje booleovsku funkciu. Tento zápis je vhodnejší pre fyzikálne zamerané úlohy, alebo pre pokročilejšie úlohy, ktoré obsahujú zložitejšie funkcie, a tie sa dajú prehľadne zobraziť logickým obvodom. Logický obvod narozdiel od predošlých reprezentácií neukazuje všetky možné kombinácie hodnôt, ale len štruktúru danej funkcie. Dá sa použiť aj pre reprezentáciu funkcie o viacerých premenných než predošlé alternatívy. Príklad zobrazenia funkcie $(A \wedge B) \vee C$ vidíme na Obrázku \ref{picture:logicalCircuit}.

\begin{figure}[ht]
	\centering
	\begin{minipage}{.5\textwidth}
		\centering
		{\includegraphics{obrazky-figures/KarnaughMap.png}}
		\caption{Karnaughova Mapa}
		\label{picture:KarnaughMap}
	\end{minipage}%
	\begin{minipage}{.5\textwidth}
		\centering
		{\includegraphics{obrazky-figures/logicalCircuit.png}}
		\caption{Logický obvod}
		\label{picture:logicalCircuit}
	\end{minipage}
\end{figure}

V technických odvetviach sa využívajú určité štandardné výrazy, ktoré sa dajú dobre využiť pri vytváraní kombinačných obvodov. Tieto výrazy sa nazývajú normálne formy a existuje ich niekoľko. Rôznymi typmi normálnych foriem sa zaoberá sekcia \ref{normal-forms}.

Pre strojovú reprezentáciu Booleovskych funkcií sa ukázali vhodné aj binárne rozhodovacie diagramy (BDD) a ich rôzne modifikácie, bude im venovaná samostatná sekcia \ref{bdds}.

\section{Normálne formy} \label{normal-forms}
Normálna forma je každý výraz v tvare:

$$T_1 \quad op \quad T_2 \quad op \quad T_3 \quad op \quad \ldots \quad op \quad T_n$$

kde množina $\{T_1,T_2,T_3 \ldots T_n\}$ sú navzájom rôzne termy rovnakého typu a $op$ je operácia v Boolovskej algebre.
Podľa typu termov a typu operácie poznáme niekoľko základných normálnych foriem. \cite{Hazewinkel-math}

\begin{itemize}
	\item disjunktívna - termy sú konjunkciou premenných a operáciou je disjunkcia
	\item konjunktívna - termy sú disjunkciou premenných a operáciou je konjunkcia  
\end{itemize}

Ak sa v každom terme v spomenutých normálnych formách vyskytuje premenná práve raz, tieto normálne formy nazývame úplná disjunktívna/konjunktívna normálna forma. 
Ak vynecháme redundantné členy, nazývame ich iredundantné normálne formy.

\section{Algebraická normálna forma}

Algebraická normálna forma (skrátene ANF) je jeden z možných spôsobov reprezentácie booleovskych funkcií. Iný názov pre zápis v ANF je aj Zhegalkinov polynóm alebo Reed-Mullerov výraz.
	
Celá ANF má hodnotu z množiny $\{0,1\}$, a skladá sa z viacerých termov, ktoré majú takisto hodnotu z množiny $\{0,1\}$. Každý term vznikol kombináciou premenných spojených operáciou AND. Termy sú spojené pomocou operácie XOR. Operácia NOT nie je v ANF povolená.
Príklad algebraickej normálnej formy: 

$$A \quad \oplus \quad B \quad \oplus \quad AB \quad \oplus \quad ABC$$

Z programátorského pohľadu môžeme hodnotu každého termu reprezentovať ako integer modulo 2. Každý term je v terminológií podľa {\color{red} ODKAZ} jednoduchým polynómom, ktorý v sebe neobsahuje koeficienty ani exponenty. Koeficienty nepotrebujeme, pretože 1 je jediný nenulový koeficient. Exponenty nie sú potrebné pretože v móde modulo 2 platí: $x^2 = x$. Preto napríklad aj zložitejší polynóm ako $3^x 2^y 5^z$ môžeme prepísať na $xyz$.
	
\vspace*{0.5\baselineskip}
Pomocou operácií $\wedge$ a $\neg$ dokážeme vytvoriť všetky ostatné operácie v Booleovskej algebre. Ďalšie operácie sú tvorené len kombináciou týchto dvoch operácií. Keďže v ANF je nie povolená operácia NOT, musíme si ju nejako vytvoriť. Negácia v ANF vzniká XORom premennej a logickej jedničky: x$\oplus$1.

%	or in standard propositional logic symbols:
	
%	${a\veebar b\veebar \left(a\wedge b\right)\veebar \left(a\wedge b\wedge c\right)} $ 

\vspace*{0.5\baselineskip}

{\color{red}
An example application is the representation of the Boolean 2-out-of-3 threshold or {\color{blue} median operation} as the Zhegalkin polynomial xy$\oplus$yz$\oplus$zx, which is 1 when at least two of the variables are 1 and 0 otherwise.

\begin{itemize}
	\item	Method of Indeterminate Coefficients
	\item 	Canonical Disjunctive Normal Form
\end{itemize}}

\section{Binárne rozhodovacie diagramy} \label{bdds}
Binárne rozhodovacie diagramy (BDD) sú triedou grafov, ktorá je prevažne využívaná ako dátová štruktúra pre reprezentáciu Booleovskych funkcií. Používajú sa na riešenie problémov ekvivalencie výrazov. Sú veľmi dôležité v oblastiach HW designu a optimalizácie.  

\vspace*{0.5\baselineskip}
Pre BDD platia tieto pravidlá:
\begin{itemize}
	\item BDD obsahuje práve 1 uzol, ktorý nemá žiadnych predchodcov - nazývame ho koreň
	\item obsahuje 1 alebo 2 uzly, ktoré nemajú ďalších nasledovníkov (označuje 0, resp. 1)
	\item všetky ostatné uzly sú označené názvom premennej a majú práve 2 nasledovníkov: {\color{red} 0-child a 1-child}, hrany vedúce k týmto nasledovníkom sú označené 0 a 1
	\item každý synovský uzol je označený buď ako 0,1 alebo premennou vyššou ako označenie otcovského uzlu.
\end{itemize}

Označenie BDD sa často používa pre označenie iného typu binárneho rozhodovacieho diagramu, a to ROBDD (Reduced Ordered BDD). V podstate sa jedná o optimalizovaný klasický BDD. Pre ROBDD platia 2 pravidlá oproti BDD:

\begin{itemize}
	\item ordered - ak sa premenné na rôznych cestách v diagramu vyskytujú všade v rovnakom poradí
	\item reduced - všetky izomorfné podgrafy sú spojené, a každý uzol, ktorého deti sú izomorfné je zanedbaný
\end{itemize} 

\chapter{Existujúce knižnice}
Existuú viaceré knižnice vytvorené za účelom manipulácie s Booleovskymi funkciami. Nasledujúca kapitola sa zaoberá niektorými vybranými, hlavne tými, ktoré využívajú binárne rozhodovacie stromy (BDD).

\section{Colorado University Decision Diagram
	Package - CUDD}
	CUDD je verejne dostupná knižnica\footnote{ \url{http://vlsi.colorado.edu/~fabio/} }, ktorej vývoj sa začal už v 70. rokoch a naďalej pokračuje.
	
	\vspace*{0.5\baselineskip}
	Balíček je možné využívať ako tzv. \textit{black box}, teda používať len exportované funkcie, ale aj ako tzv. \textit{clean box}, kde si programátor vie dodať vlastné doplňujúce funkcie.
	
	\vspace*{0.5\baselineskip}
	Je napísaná v jazyku C a poskytuje funkcie pre manipuláciu s BDD, s algebraickými rozhodovacími diagramami (ADD, MTBDD) a s diagramami s potlačenou nulou (ZDD). Takisto poskytuje možnosť prevádzať medzi jednotlivými typmi diagramov.
	
	\vspace*{0.5\baselineskip}
	CUDD využíva ukazovatele na uzly BDD. Udržuje si počítadlo referencií. Počet premenných ovplyvňuje počet tabuliek. Knižnica využíva heuristiku, ktorá sprístupní tabuľku výpočtov len vtedy, ak aspoň jeden argument má hodnotu počítadla referencií väčšiu než 1.
	
	\vspace*{0.5\baselineskip}
	V CUDD existuje veľmi efektívny správca pamäte. Garbage Collector podľa počítadla referencií maže \textit{mrtvé uzly}, teda uzly, ktoré majú 0 v počítadle referencií.
	
	Ďalšie informácie o knižnici sa dajú dohľadať v manuáli \cite{CUDD-manual}.
	
	\section{CacBDD}
	Knižnica CacBDD je verejne dostupná\footnote{ \url{http://kailesu.net/CacBDD/}} podobne ako knižnica CUDD, narozdiel od nej je ale implementovaná v jazyku C++. Je založená na prehľadávaní do hĺbky.
	 
	\vspace*{0.5\baselineskip}
	Poskytuje základné operácie pre manipuláciu s BDD. BDD uzly sú uložené v jednom poli a využíva indexy uzlov v tomto poli namiesto ukazateľov na uzly ako tomu je v CUDD. Nevyužíva počítadlo referencií na uzly. Garbace collector je volaný len ak dôjde pamäť. Funguje trošku inak ako v prípade CUDD, prechádza všetky uzly v poli, a tie na ktoré sa nikto neodkazuje a ani nie sú koreňmi, označí ako voľné uzly, nemaže ich a tým šetrí výpočtový čas. Knižnica využíva dynamické zväčšovanie tabuľky výpočtov podľa potreby, ak dôjde počet voľných miest. V knižnice je veľmi dobre implementované ukladanie medzivýsledkov, čo takisto pridáva na rýchlosti.
	
	\vspace*{0.5\baselineskip}
	Ďalšie informácie sú popísané v manuáli \cite{CacBDD-manual}, kde aj ukázané, že knižnica pracuje rýchlejšie než knižnica CUDD.

\section{BuDDy}
	Knižnica BuDDy je ďalšou knižnicou na prácu s Booleovskymi výrazmi. Je naprogramovaná v jazyku C, ale obsahuje obaľovacie C++ rozhranie pre jednoduchšiu prácu. 
	
	\vspace*{0.5\baselineskip}
	Obsahuje vlastný Garbage Collector, cache pamäť na uchovanie medzivýsledkov. Takmer každé nastavenie činnosti sa dá ručne prenastaviť, ale obsahuje aj základné nastavenia pre užívateľov, ktorí sa v nastaveniach hrabať nechcú.

	\vspace*{0.5\baselineskip}
	Knižnica obsahuje veľké množstvo funkcií a operácií, ktoré sa dajú použiť na prácu s Booleovskymi funkciami. Všetky výsledky v BuDDy sú reprezentované vektormi, a tým pádom sa s nimi v C++ ľahšie manipuluje.
	
	\section{BCL - Class Library for Boolean Function Manipulation}
	Knižnica pre manipuláciu s Booleovskymi funkciami vytvorená v jazyku C\#, je vhodná pre využitie v jazykoch z rodiny .NET Framework. 
	
	\vspace*{0.5\baselineskip}
	Obsahuje viaceré interné reprezentácie Booleovskych funkcií, ako sú pravdivostné tabuľky, booleovske výrazy a BDD.  Každá z reprezentácií obsahuje metódy na zjednodušenie funkcie, vytvorenie novej funkcie aplikovaním operátoru na 2 funkcie, na nahradenie premennej konštantou a pre nahradenie premennej inou funkciou.
	
	\vspace*{0.5\baselineskip}
	Knižnica sa využíva hlavne na výskumné účely, pretože obsahuje užitočné funkcie na určenie Shannonovho rozvoja, zistenie linearity a monotónnosti funkcie a mnohé ďalšie.
	Takisto obsahuje metódy konverzie medzi reprezentáciami, okrem iných aj konvertor z pravdivostnej tabuľky na ANF, DNF, CNF a BDD.
	
	
	\section{CORAL}
	Knižnica napísaná v jazyku C++, ktorá bola zamýšľaná na použitie v logických programovacích jazykoch, ale aj v iných. Podobne ako ostatné knižnice využíva ROBDD - Reduced Ordered BDD. Knižnica je zameraná hlavne na pamäťovú efektivitu a na optimalizáciu.
	
	\section{BDD}
	Knižnica napísaná v C, primárne zameraná na operačné systémy UNIX, pre prácu mimo UNIX je potrebné upraviť správcu pamäte. Knižnica je rozsahovo veľmi malá\footnote{ \url{http://www.cs.cmu.edu/afs/cs/project/modck/pub/www/bdd.html} }. 
	
	\vspace*{0.5\baselineskip}
	Obsahuje nástroje na sekvenčné overovanie, cache pamäť na ukladanie výsledkov, kam sa ukladajú úplne všetky medzivýsledky, kvantifikácie viacerých premenných a substitúcie. Okrem toho obsahuje nástroje na analýzu BDD, napríklad histogram, možnosť uloženia BDD do súborov.
	
	\vspace*{0.5\baselineskip}
	Garbage collector funguje na báze počítadla referencií alebo na princípe "zmaž všetko okrem". Takisto používateľ dokáže nastaviť limit na počet uzlov, operácie samé zmažú pamäť ak by museli prekročiť tento limit. Knižnica poskytuje aj možnosť dynamického preusporiadania premenných.

\section{PPBF BDD - Parallel partial breadth-first expansion}
	Knižnica\footnote{\url{http://www.cs.cmu.edu/~bwolen/software/} } pre multiprocesorové paralelné spracovanie BDD. Na prácu potrebuje zdieľanú pamäť. Poskytuje operácie nad kombinačnými obvodmi. 
	
\chapter{Návrh}
\chapter{Implementácia}
\chapter{Vyhodnotenie}
\chapter{Záver}
	
	
\chapter{TODO}	

boolean satisfability SAT, picosat

	ODKAZY:
				
	Pravdivostne tabulky:
	\begin{itemize}
		\item Georg Henrik von Wright (1955). "Ludwig Wittgenstein, A Biographical Sketch". The Philosophical Review. 64 (4): 527–545 (p. 532, note 9). doi:10.2307/2182631. JSTOR 2182631.
		\item Emil Post (July 1921). "Introduction to a general theory of elementary propositions". American Journal of Mathematics. 43 (3): 163–185. doi:10.2307/2370324. JSTOR 2370324.
		\item Ludwig Wittgenstein (1922) Tractatus Logico-Philosophicus \url{http://www.gutenberg.org/ebooks/5740?msg=welcome_stranger}
		\item Anellis, Irving H. (2012). "Peirce's Truth-functional Analysis and the Origin of the Truth Table". History and Philosophy of Logic. 33: 87–97. doi:10.1080/01445340.2011.621702.
	\end{itemize}
BDD:
	\begin{itemize}
		\item  Graph-Based Algorithms for Boolean Function Manipulation, Randal E. Bryant, 1986
		\item C. Y. Lee. "Representation of Switching Circuits by Binary-Decision Programs". Bell System Technical Journal, 38:985–999, 1959.
		\item Sheldon B. Akers. Binary Decision Diagrams, IEEE Transactions on Computers, C-27(6):509–516, June 1978.
		\item Raymond T. Boute, "The Binary Decision Machine as a programmable controller". EUROMICRO Newsletter, Vol. 1(2):16–22, January 1976.
		\item Randal E. Bryant. "Graph-Based Algorithms for Boolean Function Manipulation". IEEE Transactions on Computers, C-35(8):677–691, 1986.
		\item R. E. Bryant, "Symbolic Boolean Manipulation with Ordered Binary Decision Diagrams", ACM Computing Surveys, Vol. 24, No. 3 (September, 1992), pp. 293–318.
		\item Karl S. Brace, Richard L. Rudell and Randal E. Bryant. "Efficient Implementation of a BDD Package". In Proceedings of the 27th ACM/IEEE Design Automation Conference (DAC 1990), pages 40–45. IEEE Computer Society Press, 1990.
		\item http://scpd.stanford.edu/knuth/index.jsp
		\item R.M. Jensen. "CLab: A C+ + library for fast backtrack-free interactive product configuration". Proceedings of the Tenth International Conference on Principles and Practice of Constraint Programming, 2004.
		\item H.L. Lipmaa. "First CPIR Protocol with Data-Dependent Computation". ICISC 2009.
		\item Beate Bollig, Ingo Wegener. Improving the Variable Ordering of OBDDs Is NP-Complete, IEEE Transactions on Computers, 45(9):993–1002, September 1996.
		\item Detlef Sieling. "The nonapproximability of OBDD minimization." Information and Computation 172, 103–138. 2002.
		\item Rice, Michael. "A Survey of Static Variable Ordering Heuristics for Efficient BDD/MDD Construction" (PDF).
		\item Philipp Woelfel. "Bounds on the OBDD-size of integer multiplication via universal hashing." Journal of Computer and System Sciences 71, pp. 520-534, 2005.
		\item Richard J. Lipton. "BDD's and Factoring". Gödel's Lost Letter and P=NP, 2009.
		\item Andersen, H. R. (1999). "An Introduction to Binary Decision Diagrams" (PDF). Lecture Notes. IT University of Copenhagen.
	\end{itemize}
	ANF:
	\begin{itemize}
		\item Bell, Eric (1927). "Arithmetic of Logic". Transactions of the American Mathematical Society. Transactions of the American Mathematical Society, Vol. 29, No. 3. 29 (3): 597–611. doi:10.2307/1989098. JSTOR 1989098.
		\item Gindikin, S.G. (1972). Algebraic Logic. Moscow: Nauka (English translation Springer-Verlag 1985). ISBN 0-387-96179-8.
		\item Stone, Marshall (1936). "The Theory of Representations for Boolean Algebras". Transactions of the American Mathematical Society. Transactions of the American Mathematical Society, Vol. 40, No. 1. 40 (1): 37–111. doi:10.2307/1989664. ISSN 0002-9947. JSTOR 1989664.
		\item Zhegalkin, Ivan Ivanovich (1927). "On the Technique of Calculating Propositions in Symbolic Logic". Matematicheskii Sbornik. 43: 9–28.
	\end{itemize}
KNF:
	\begin{itemize}
		\item Paul Jackson, Daniel Sheridan: Clause Form Conversions for Boolean Circuits. In: Holger H. Hoos, David G. Mitchell (Eds.): Theory and Applications of Satisfiability Testing, 7th International Conference, SAT 2004, Vancouver, BC, Canada, May 10–13, 2004, Revised Selected Papers. Lecture Notes in Computer Science 3542, Springer 2005, pp. 183–198
		\item G.S. Tseitin: On the complexity of derivation in propositional calculus. In: Slisenko, A.O. (ed.) Structures in Constructive Mathematics and Mathematical Logic, Part II, Seminars in Mathematics (translated from Russian), pp. 115–125. Steklov Mathematical Institute (1968)
	\end{itemize}
DNF:
\begin{itemize}
	\item  B.A. Davey and H.A. Priestley (1990). Introduction to Lattices and Order. Cambridge Mathematical Textbooks. Cambridge University Press.
\end{itemize}
Majority function:
\begin{itemize}
	\item Knuth, Donald E. (2008). Introduction to combinatorial algorithms and Boolean functions. The Art of Computer Programming. 4a. Upper Saddle River, NJ: Addison-Wesley. pp. 64–74. ISBN 0-321-53496-4.
\end{itemize}
Reed Muller:
\begin{itemize}
	\item Kebschull, U. and Rosenstiel, W., Efficient graph-based computation and manipulation of functional decision diagrams, Proceedings 4th European Conference on Design Automation, 1993, pp. 278–282
\end{itemize}