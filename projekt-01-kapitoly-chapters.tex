\chapter{Úvod}
Booleovské funkcie majú v dnešnej dobe veľké využitie. V matematike sa využívajú na určenie pravdivostnej hodnoty výrokov, v elektrotechnike na vytváranie kombinačných obvodov, takisto ich môžeme vidieť na pozadí teórie hier či v legislatíve.

TODO
\begin{itemize}
	\item popisat motivaciu k projektu
	\item aktualne pouzivane reprezentacie
	\item existujuce riesenie
	\item kratky popis vytvoreneho riesenia
\end{itemize}
\chapter{Booleovske funkcie}
Táto kapitola popisuje čo sú to booleovske funkcie, čo môžu obsahovať a ako ich môžeme prehľadne znázorniť. Ďalej sú popísané rôzne normalizované formy zápisu booleovskych funkcií, primárne algebraická normálna forma. 

\section{Definícia booleovskej funkcie}
Ako uvádza Crama \cite{Crama-bool}, booleovská funkcia je každá funkcia $f: \mathcal{B}^{n} \rightarrow \mathcal{B}$, kde $\mathcal{B}$ je množina $\{0,1\}$, $n$ je kladné prirodzené číslo, a $\mathcal{B}^{n}$ označuje $n$-násobný kartézsky súčin množiny $\mathcal{B}$ samej so sebou. Každý bod funkcie $X^*$ = ($x_1,x_2, \ldots, x_n$) naberá hodnotu buď 0 alebo 1.

Celkový počet booleovskych funkcií pre $n$ premenných je $2^{2^n}$. Je to dané tým, že všetkých možných kombinácií vstupných parametrov je ($2^n$) a parametre môžu mať hodnotu z $\{0,1\}$. Počet možných booleovských funkcií pre niektoré hodnoty $n$ sa nachádza v Tabuľke {\ref{table:functionCountExample}}.

\begin{table}[h]
	\centering
	\begin{tabular}{|r|l|} \hline
		n & počet funkcií \\ \hline
		1 & 4   \\
		2 & 16  \\
		3 & 256 \\
		5 & 4.29497 $\times 10^9$ \\ \hline
	\end{tabular}
	\caption{Počet booleovských funkcií pre vybrané hodnoty $n$}
	\label{table:functionCountExample}
\end{table}

V mnohých aplikáciách sa namiesto hodnôt \{0,1\} používa iná dvojica, napríklad \{true,false\}, \{on,off\}, \{áno,nie\}, vždy to ale označuje opačné hodnoty. 
Množina $\mathcal{B}$ spolu so základnými booleovskymi operáciami konjunkciou, disjunkciou a negáciou tvorí Booleovsku algebru.

\section{Reprezentácia booleovskych funkcií}
Booleovske funkcie môžu byť vyjadrené rôznymi spôsobmi. Záleží hlavne na tom, čo plánujeme s danou funkciou robiť. Niektoré zápisy sú vhodnejšie na matematické výpočty, iné zase na prehľadné prezeranie dát. 

\vspace*{0.5\baselineskip}
Prvým možným zápisom je tzv. pravdivostná tabuľka. Je to tabuľka, v ktorej na každom riadku je hodnota funkcie pri inú kombináciu vstupných hodnôt funkcie. Pravdivostné tabuľky majú dobré využitie pre funkcie do 4-5 parametrov. Pre vyšší počet parametrov sa stávajú neprehľadnými pre vysoký počet možných kombinácií. Príklad pravdivostnej tabuľky pre 2 vstupné hodnoty sa nachádza v Tabuľke \ref{table:truthTable}. 

\begin{table}[h]
	\centering
	\begin{tabular}{|c|c|} \hline
		$(x_1,x_2)$ & $f(x_1,x_2)$ \\ \hline
		$(0,0)$ & 0 \\
		$(0,1)$ & 1 \\
		$(1,0)$ & 1 \\
		$(1,1)$ & 0 \\ \hline
	\end{tabular}
	\caption{Pravdivostná tabuľka}
	\label{table:truthTable}
\end{table}

Upravenou formou pravdivostnej tabuľky je Karnaughova mapa. Je to forma zápisu ktorá prevádza n-rozmernú booleovsku funkciu do 2-rozmernej. Využíva sa hlavne pri minimalizácii funkcií. Príklad sa nachádza na Obrázku \ref{picture:KarnaughMap}

\begin{figure}[ht]
	\centering
	\scalebox{0.90}
	{\includegraphics{obrazky-figures/KarnaughMap.png}}
	\caption{Karnaughova Mapa}
	\label{picture:KarnaughMap}
\end{figure}

Ďalším zo zápisov je logický obvod. Ide o schému, ktorá graficky zobrazuje booleovsku funkciu. Tento zápis je vhodnejší pre fyzikálne zamerané úlohy, alebo pre pokročilejšie úlohy, ktoré obsahujú zložitejšie funkcie, a tie sa dajú prehľadne zobraziť logickým obvodom. Príklad zobrazenia funkcie $(A \wedge B) \vee C$ vidíme na Obrázku \ref{picture:logicalCircuit}

\begin{figure}[ht]
	\centering
	\scalebox{0.80}
	{\includegraphics{obrazky-figures/logicalCircuit.png}}
	\caption{Logický obvod}
	\label{picture:logicalCircuit}
\end{figure}

V technických odvetviach sa využívajú určité štandardné výrazy, ktoré sa dajú dobre využiť pri vytváraní kombinačných obvodov. Tieto výrazy sa nazývajú normálne formy a existuje ich niekoľko. Rôznymi typmi normálnych foriem sa zaoberá nasledujúca sekcia.

\section{Normálne formy}
Normálna forma je každý výraz v tvare:

$$T_1 \quad op \quad T_2 \quad op \quad T_3 \quad op \quad \ldots \quad op \quad T_n$$

kde množina $\{T_1,T_2,T_3 \ldots T_n\}$ sú navzájom rôzne termy rovnakého typu a $op$ je operácia v Boolovskej algebre.
Podľa typu termov a typu operácie poznáme niekoľko základných normálnych foriem.

\begin{itemize}
	\item disjunktívna - termy sú konjunkciou premenných a operáciou je disjunkcia
	\item konjunktívna - termy sú disjunkciou premeenných a operáciou je konjunkcia  
\end{itemize}

Ak sa v každom terme v spomenutých normálnych formách vyskytuje premenná práve raz, tieto normálne formy nazývame úplná disjunktívna/konjunktívna normálna forma. 
Ak vynecháme redundantné členy, nazývame ich iredundantné normálne formy.

\section{Algebraická normálna forma}

Algebraická normálna forma (skrátene ANF) je jeden z možných spôsobov reprezentácie booleovskych funkcií. Iný názov pre zápis v ANF je aj Zhegalkinov polynóm alebo Reed-Mullerov výraz.
	
Celá ANF má hodnotu z množiny $\{0,1\}$, a skladá sa z viacerých termov, ktoré majú takisto hodnotu z množiny $\{0,1\}$. Každý term vznikol kombináciou premenných spojených operáciou AND. Termy sú spojené pomocou operácie XOR. Operácia NOT nie je v ANF povolená.
Príklad algebraickej normálnej formy: 

$$A \quad \oplus \quad B \quad \oplus \quad AB \quad \oplus \quad ABC$$

Z programátorského pohľadu môžeme hodnotu každého termu reprezentovať ako integer modulo 2. Každý term je v terminológií podľa {\color{red} ODKAZ} jednoduchým polynómom, ktorý v sebe neobsahuje koeficienty ani exponenty. Koeficienty nepotrebujeme, pretože 1 je jediný nenulový koeficient. Exponenty nie sú potrebné pretože v móde modulo 2 platí: $x^2 = x$. Preto napríklad aj zložitejší polynóm ako $3^x 2^y 5^z$ môžeme prepísať na $xyz$.
	
\vspace*{0.5\baselineskip}
Pomocou operácií $\wedge$ a $\neg$ dokážeme vytvoriť všetky ostatné operácie v Booleovskej algebre. Ďalšie operácie sú tvorené len kombináciou týchto dvoch operácií. Keďže v ANF je nie povolená operácia NOT, musíme si ju nejako vytvoriť. Negácia v ANF vzniká XORom premennej a logickej jedničky: x$\oplus$1.

%	or in standard propositional logic symbols:
	
%	${a\veebar b\veebar \left(a\wedge b\right)\veebar \left(a\wedge b\wedge c\right)} $ 

\vspace*{0.5\baselineskip}

{\color{red}
An example application is the representation of the Boolean 2-out-of-3 threshold or {\color{blue} median operation} as the Zhegalkin polynomial xy$\oplus$yz$\oplus$zx, which is 1 when at least two of the variables are 1 and 0 otherwise.

\begin{itemize}
	\item	Method of Indeterminate Coefficients
	\item 	Canonical Disjunctive Normal Form
\end{itemize}}

\section{Binárne rozhodovacie diagramy}
TODO

\chapter{Existujúce knižnice}
Existuú viaceré knižnice vytvorené za účelom manipulácie s Booleovskymi funkciami. Nasledujúca kapitola sa zaoberá niektorými vybranými, hlavne tými, ktoré využívajú binárne rozhodovacie stromy (BDD).

\section{Colorado University Decision Diagram
	Package - CUDD}
	CUDD je verejne dostupná knižnica\footnote{ \url{http://vlsi.colorado.edu/~fabio/} }, ktorej vývoj sa začal už v 70. rokoch a naďalej pokračuje.
	
	\vspace*{0.5\baselineskip}
	Balíček je možné využívať ako tzv. \textit{black box}, teda používať len exportované funkcie, ale aj ako tzv. \textit{clean box}, kde si programátor vie dodať vlastné doplňujúce funkcie.
	
	\vspace*{0.5\baselineskip}
	Je napísaná v jazyku C a poskytuje funkcie pre manipuláciu s BDD, s algebraickými rozhodovacími diagramami (ADD, MTBDD) a s diagramami s potlačenou nulou (ZDD). Takisto poskytuje možnosť prevádzať medzi jednotlivými typmi diagramov.
	
	\vspace*{0.5\baselineskip}
	CUDD využíva ukazovatele na uzly BDD. Udržuje si počítadlo referencií. Počet premenných ovplyvňuje počet tabuliek. Knižnica využíva heuristiku, ktorá sprístupní tabuľku výpočtov len vtedy, ak aspoň jeden argument má hodnotu počítadla referencií väčšiu než 1.
	
	\vspace*{0.5\baselineskip}
	V CUDD existuje veľmi efektívny správca pamäte. Garbage Collector podľa počítadla referencií maže \textit{mrtvé uzly}, teda uzly, ktoré majú 0 v počítadle referencií.
	
	Ďalšie informácie o knižnici sa dajú dohľadať v manuáli \cite{CUDD-manual}.
	
	\section{CacBDD}
	Knižnica CacBDD je verejne dostupná\footnote{ \url{http://kailesu.net/CacBDD/}} podobne ako knižnica CUDD, narozdiel od nej je ale implementovaná v jazyku C++. Je založená na prehľadávaní do hĺbky.
	 
	\vspace*{0.5\baselineskip}
	Poskytuje základné operácie pre manipuláciu s BDD. BDD uzly sú uložené v jednom poli a využíva indexy uzlov v tomto poli namiesto ukazateľov na uzly ako tomu je v CUDD. Nevyužíva počítadlo referencií na uzly. Garbace collector je volaný len ak dôjde pamäť. Funguje trošku inak ako v prípade CUDD, prechádza všetky uzly v poli, a tie na ktoré sa nikto neodkazuje a ani nie sú koreňmi, označí ako voľné uzly, nemaže ich a tým šetrí výpočtový čas. Knižnica využíva dynamické zväčšovanie tabuľky výpočtov podľa potreby, ak dôjde počet voľných miest. V knižnice je veľmi dobre implementované ukladanie medzivýsledkov, čo takisto pridáva na rýchlosti.
	
	\vspace*{0.5\baselineskip}
	Ďalšie informácie sú popísané v manuáli \cite{CacBDD-manual}, kde aj ukázané, že knižnica pracuje rýchlejšie než knižnica CUDD.


**********************************************************

\begin{itemize}
	\item 	BuDDy - http://www.drdobbs.com/the-buddy-library-boolean-expressions/184401847
	\item	nejaka C\# kniznica - http://dispatcher.swu.bg/BCL/
	\item CORAL - https://www.yumpu.com/en/document/view/20494846/a-modern-c-library-for-the-manipulation-of-boolean-functions
\end{itemize}
	{\color{red} rozpisat na podkapitoly}
	
	
\chapter{TODO}	
	ODKAZY:
	
	Boolovske funkcie:
	\begin{itemize}
		\item Crama, Y; Hammer, P. L. (2011), Boolean Functions, Cambridge University Press.
		\item Hazewinkel, Michiel, ed. (2001), "Boolean function", Encyclopedia of Mathematics, Springer, ISBN 978-1-55608-010-4
		\item Janković, Dragan; Stanković, Radomir S.; Moraga, Claudio (November 2003). "Arithmetic expressions optimisation using dual polarity property" (PDF). Serbian Journal of Electrical Engineering. 1 (71-80, number 1). Retrieved 2015-06-07.
		\item Mano, M. M.; Ciletti, M. D. (2013), Digital Design, Pearson.
					
	\end{itemize}
	Pravdivostne tabulky:
	\begin{itemize}
		\item Georg Henrik von Wright (1955). "Ludwig Wittgenstein, A Biographical Sketch". The Philosophical Review. 64 (4): 527–545 (p. 532, note 9). doi:10.2307/2182631. JSTOR 2182631.
		\item Emil Post (July 1921). "Introduction to a general theory of elementary propositions". American Journal of Mathematics. 43 (3): 163–185. doi:10.2307/2370324. JSTOR 2370324.
		\item Ludwig Wittgenstein (1922) Tractatus Logico-Philosophicus \url{http://www.gutenberg.org/ebooks/5740?msg=welcome_stranger}
		\item Anellis, Irving H. (2012). "Peirce's Truth-functional Analysis and the Origin of the Truth Table". History and Philosophy of Logic. 33: 87–97. doi:10.1080/01445340.2011.621702.
	\end{itemize}
	Zhegalkin:
	\begin{itemize}
		\item Bell, Eric (1927). "Arithmetic of Logic". Transactions of the American Mathematical Society. Transactions of the American Mathematical Society, Vol. 29, No. 3. 29 (3): 597–611. doi:10.2307/1989098. JSTOR 1989098.
		\item Gindikin, S.G. (1972). Algebraic Logic. Moscow: Nauka (English translation Springer-Verlag 1985). ISBN 0-387-96179-8.
		\item Stone, Marshall (1936). "The Theory of Representations for Boolean Algebras". Transactions of the American Mathematical Society. Transactions of the American Mathematical Society, Vol. 40, No. 1. 40 (1): 37–111. doi:10.2307/1989664. ISSN 0002-9947. JSTOR 1989664.
		\item Zhegalkin, Ivan Ivanovich (1927). "On the Technique of Calculating Propositions in Symbolic Logic". Matematicheskii Sbornik. 43: 9–28.
	\end{itemize}
KOnjunktivna NF:
	\begin{itemize}
		\item Paul Jackson, Daniel Sheridan: Clause Form Conversions for Boolean Circuits. In: Holger H. Hoos, David G. Mitchell (Eds.): Theory and Applications of Satisfiability Testing, 7th International Conference, SAT 2004, Vancouver, BC, Canada, May 10–13, 2004, Revised Selected Papers. Lecture Notes in Computer Science 3542, Springer 2005, pp. 183–198
		\item G.S. Tseitin: On the complexity of derivation in propositional calculus. In: Slisenko, A.O. (ed.) Structures in Constructive Mathematics and Mathematical Logic, Part II, Seminars in Mathematics (translated from Russian), pp. 115–125. Steklov Mathematical Institute (1968)
	\end{itemize}
DNF:
\begin{itemize}
	\item  B.A. Davey and H.A. Priestley (1990). Introduction to Lattices and Order. Cambridge Mathematical Textbooks. Cambridge University Press.
\end{itemize}
Majority function:
\begin{itemize}
	\item Knuth, Donald E. (2008). Introduction to combinatorial algorithms and Boolean functions. The Art of Computer Programming. 4a. Upper Saddle River, NJ: Addison-Wesley. pp. 64–74. ISBN 0-321-53496-4.
\end{itemize}
Reed Muller:
\begin{itemize}
	\item Kebschull, U. and Rosenstiel, W., Efficient graph-based computation and manipulation of functional decision diagrams, Proceedings 4th European Conference on Design Automation, 1993, pp. 278–282
\end{itemize}
\chapter{Záver}