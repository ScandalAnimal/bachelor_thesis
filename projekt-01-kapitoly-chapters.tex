\chapter{Úvod}
TODO
\chapter{Booleovske funkcie}
Booleovské funkcie majú v dnešnej dobe veľké využitie. V matematike sa využívajú na určenie pravdivostnej hodnoty výrokov, v elektrotechnike na vytváranie kombinačných obvodov, takisto ich môžeme vidieť na pozadí teórie hier či v legislatíve.

Existujú viaceré možnosti znázornenia booleovskej funkcie, napríklad pomocou pravdivostnej tabuľky, booleovskeho výrazu, Karnaughovej mapy, obvodu alebo grafu. 

\section{Definícia booleovskej funkcie}
Ako uvádza Crama \cite{Crama-bool}, booleovská funkcia je každá funkcia $f: \mathcal{B}^{n} \rightarrow \mathcal{B}$, kde $\mathcal{B}$ je množina $\{0,1\}$, $n$ je kladné prirodzené číslo, a $\mathcal{B}^{n}$ označuje $n$-násobný kartézsky súčin množiny $\mathcal{B}$ samej so sebou. Každý bod funkcie $X^*$ = ($x_1,x_2, \ldots, x_n$) naberá hodnotu buď 0 alebo 1.

Celkový počet booleovskych funkcií pre $n$ premenných je $2^{2^n}$. Je to dané tým, že na každom riadku sa môže vyskytnúť jedna z hodnôt $\{0,1\}$, a riadkov je toľko, koľko je všetkých možných kombinácií vstupných parametrov ($2^n$).  Pravdivostné tabuľky majú dobré využitie pre funkcie do 4-5 parametrov. Pre vyšší počet parametrov sa stávajú neprehľadnými. Počet možných booleovských funkcií pre niektoré hodnoty $n$ sa nachádza v Tabuľke {\ref{table:functionCountExample}}.

\begin{table}[h]
	\centering
	\begin{tabular}{|r|l|} \hline
		n & počet funkcií \\ \hline
		1 & 4   \\
		2 & 16  \\
		3 & 256 \\
		5 & 4.29497 $\times 10^9$ \\ \hline
	\end{tabular}
	\caption{Počet booleovských funkcií pre vybrané hodnoty $n$}
	\label{table:functionCountExample}
\end{table}

V mnohých aplikáciách sa namiesto hodnôt \{0,1\} používa iná dvojica, napríklad \{true,false\}, \{on,off\}, \{áno,nie\}, vždy to ale označuje opačné hodnoty. 
Množina $\mathcal{B}$ spolu so základnými booleovskymi operáciami konjunkciou, disjunkciou a negáciou tvorí Booleovsku algebru {\color{red} ODKAZ}.

\section{Reprezentácia booleovskych funkcií (zmeniť názov)}

{\color{red}
popis možností ako pracovať s booleovskymi funkciami , predbežne BDD, logicky obvod
}

\vspace*{0.5\baselineskip}
Pre definíciu konkrétnej booleovskej funkcie sa najčastejšie využíva tzv. pravdivostná tabuľka. Je to tabuľka, v ktorej na každom riadku je hodnota funkcie pri inú kombináciu vstupných hodnôt funkcie. Príklad pravdivostnej tabuľky pre 2 vstupné hodnoty sa nachádza v Tabuľke \ref{table:truthTable}. 

\begin{table}[h]
	\centering
	\begin{tabular}{|c|c|} \hline
		$(x_1,x_2)$ & $f(x_1,x_2)$ \\ \hline
		$(0,0)$ & 0 \\
		$(0,1)$ & 1 \\
		$(1,0)$ & 1 \\
		$(1,1)$ & 0 \\ \hline
	\end{tabular}
	\caption{Pravdivostná tabuľka}
	\label{table:truthTable}
\end{table}


\section{Algebraická normálna forma}

Algebraická normálna forma (skrátene ANF) je jeden z možných spôsobov reprezentácie booleovskych funkcií. Iný názov pre zápis v ANF je aj Zhegalkinov polynóm {\color{red} ODKAZ} alebo Reed-Mullerov výraz {\color{red} ODKAZ}.
	
Celá ANF má hodnotu 1 alebo 0, ale skladá sa z viacerých termov {\color{red} ODKAZ}, ktoré majú takisto hodnotu 1 alebo 0. Každý term vznikol kombináciou premenných spojených operáciou AND. Termy sú spojené pomocou operácie XOR. Operácia NOT nie je v ANF povolená.

	Priklad :
{\color{green}
	a $\oplus$ b $\oplus$ ab $\oplus$ abc}

Z programátorského pohľadu môžeme hodnotu každého termu reprezentovať ako integer mod 2. Každý term je v terminológií podľa {\color{red} ODKAZ na polynómy} jednoduchým polynómom, ktorý v sebe neobsahuje koeficienty ani exponenty. Koeficienty nepotrebujeme pretože 1 je jediný nenulový koeficient. Exponenty nie sú potrebné pretože v móde mod 2 platí: $x^2$ = x. Preto napríklad aj zložitejší polynóm ako $3^x2^y5^z$ môžeme prepísať na xyz.
	
Pomocou operácií $\wedge$ a $\neg$ dokážeme vytvoriť všetky ostatné operácie v Booleovskej algebre. Ďalšie operácie sú tvorené len kombináciou týchto dvoch operácií. Keďže v ANF je nie povolená operácia NOT, musíme si ju nejako vytvoriť. Negácia v ANF vzniká XORom premennej a logickej jedničky: x$\oplus$1.

%	or in standard propositional logic symbols:
	
%	${a\veebar b\veebar \left(a\wedge b\right)\veebar \left(a\wedge b\wedge c\right)} $ 

{\color{red}
An example application is the representation of the Boolean 2-out-of-3 threshold or {\color{blue} median operation} as the Zhegalkin polynomial xy$\oplus$yz$\oplus$zx, which is 1 when at least two of the variables are 1 and 0 otherwise.

There are three known methods generally used for the computation of the Zhegalkin polynomial.

\begin{itemize}
	\item	Using the Method of Indeterminate Coefficients
	\item 	By constructing the Canonical Disjunctive Normal Form
	\item	By using tables
\end{itemize}}
{\color{blue} DOVYSVETLIT vsetky 3}

\chapter{Existujúce knižnice}

\begin{itemize}
	\item	CUDD - http://vlsi.colorado.edu/~fabio/CUDD/cudd.pdf
	\item 	CacBDD - http://www.kailesu.net/CacBDD/CacBDD.pdf
	\item 	BuDDy - http://www.drdobbs.com/the-buddy-library-boolean-expressions/184401847
	\item	nejaka C\# kniznica - http://dispatcher.swu.bg/BCL/
	\item CORAL - https://www.yumpu.com/en/document/view/20494846/a-modern-c-library-for-the-manipulation-of-boolean-functions
\end{itemize}
	{\color{red} rozpisat na podkapitoly}

\chapter{Záver}